% TEMPLATE for Usenix papers, specifically to meet requirements of
%  USENIX '05
% originally a template for producing IEEE-format articles using LaTeX.
%   written by Matthew Ward, CS Department, Worcester Polytechnic Institute.
% adapted by David Beazley for his excellent SWIG paper in Proceedings,
%   Tcl 96
% turned into a smartass generic template by De Clarke, with thanks to
%   both the above pioneers
% use at your own risk.  Complaints to /dev/null.
% make it two column with no page numbering, default is 10 point

% Munged by Fred Douglis <douglis@research.att.com> 10/97 to separate
% the .sty file from the LaTeX source template, so that people can
% more easily include the .sty file into an existing document.  Also
% changed to more closely follow the style guidelines as represented
% by the Word sample file. 

% Note that since 2010, USENIX does not require endnotes. If you want
% foot of page notes, don't include the endnotes package in the 
% usepackage command, below.

% This version uses the latex2e styles, not the very ancient 2.09 stuff.
\documentclass[letterpaper,twocolumn,10pt]{article}
\usepackage{usenix,epsfig,endnotes}
\usepackage{color}
\usepackage{colortbl}

\newcommand{\jc}[1]{{\footnotesize\color{red}[JC: #1]}}


\begin{document}

%don't want date printed
\date{}

%make title bold and 14 pt font (Latex default is non-bold, 16 pt)
\title{SDCDN: A Control Plane for Content Delivery Networks}

%for single author (just remove % characters)
\author{
{\rm Your N.\ Here}\\
Your Institution
\and
{\rm Second Name}\\
Second Institution
% copy the following lines to add more authors
% \and
% {\rm Name}\\
%Name Institution
} % end author

\maketitle

% Use the following at camera-ready time to suppress page numbers.
% Comment it out when you first submit the paper for review.
\thispagestyle{empty}


\subsection*{Abstract}
Content delivery networks (CDN) and overlay networks count for most of today's Internet traffic and have large impact on performance of applications (e.g., video streaming and web) that rely on them. 
Traditionally, these networks are based on distributed, rather than centralized protocols, which works more resilient and responsive in large geo-distribuetd networks. However, the trend of growing traffic volume and demands for high quality experience and that of global control planes for application call for a logically centralized architecture of CDNs and overlay networks at large where servers are coordinated in a global scale. 

To address this, we presents software-defined CDN (SDCDN), a logically centralized architecture for CDNs and, in particular, Live streaming. It is similar in spirit to software-defined networks (SDN), but differs in several key aspects. As a design choice, we argue that neithor fully centralization (i.e, SDN) nor fully decentralization (i.e., current CDN) is most desirable for CDN control plane. Instead, a careful division of functionalities between distributed servers and centralized controllers is needed to achieve both performance benefits of global coordination and faster recovery from failures. 

\section{Introduction}

\begin{itemize}
	\item CDNs are expected to carry more traffic. 
	\item Today's CDN architecture is highly distributed. With expended capacity it is also needed to have a globally coordinated control plane. The CDN performance will benefit from such control plane in three aspects: coordinated resource allocation and recovery from failure and simplified optimization for application goals. 
	\item Meanwhile, it has two fundamental challenges resulting from its overlay nature.
	\begin{itemize}
		\item Long propagation delay
		\item Multiple controller consistency
	\end{itemize}
	\item We present SDCDN that addresses the challenges by leveraging a unique opportunity of overlay networks -- flexibility of individual node. Instead of implementing a fully centralized control plane or pushing all control functionalities to centralized controllers, our control plane resides in both controller, called global control process, and distributed nodes called local control process. The two control processes run in different time scales.
\end{itemize}


\section{Motivation}
\subsection{Why centralization?}
\begin{itemize}
	\item Like in traditional SDN argument, a centralized control platform enables the development of flexible, reliable and feature-richer network control plane.
	\item Centralization enables global coordination of resources to optimize for complex application goal. Use example.
	\item Centralization enables better recovery and convergence in disruptive/congested networks. Use example.
\end{itemize}

\subsection{Why not fully centralized?}
\begin{itemize}
	\item Longer propagation delay than in LAN/WAN. \jc{need measurement data to support this.}
	\item Long delay to get a converged network view under large-area disruption/congestion. \jc{need measurement data to support this.}
\end{itemize}

\subsection{Two level control plane design}
Present our design of control plane that runs both in local node and controller.


\section{Design}

\subsection{Components}
Overview of whole system and functionalities of each component. 

\subsection{Interfaces}
Introduce forwarding table (between data plane and control plane), hint (within control plane between controller and nodes), NIB (between discovery modules and control plane)

\subsection{Flexible Data plane}
Our data plane can be based on existing proxy servers (e.g., Apache server) used by production networks. What flexibility we need from data plane.
\begin{itemize}
	\item Understand and execute based on the forwarding table.
	\item Provide information for in-band passive measurement.
\end{itemize}

\subsection{Discovery}
Discovery consists of local process running on each node to output local NIB, and global process that combines the local NIBs to output a global NIB.

\begin{itemize}
	\item how to compute global NIB from local NIBs
	\item when to inform global control process of a change in global NIB to prevent frequent recalculating (when it is converged?) \jc{need some idea...}
\end{itemize}

\subsection{Control plane}
The control plane design has three parts: global control process, local control process, and merging algorithm.

Global control process: globally coordinated resource share. output global hint based on application-level goals. Coarse time granular.

Local control process: detect connection failure/congestion from local NIB and output local hint.

Merging algorithm: merge global hint from multiple controllers and local hint. need to meet requirements: (i) correctness, and (ii) prefer local hint when local hint strongly suggests current forwarding table is hopeless.


\begin{comment}
\section{Fault tolerance}

We consider two types of failure models. 

\section{Controller consistency}

\begin{itemize}
	\item Inconsistency of multiple versions of same controller: formulate the problem, introduce the solution
	\item Inconsistency of different controllers: formulate the problem, introduce the solution
\end{itemize}
\end{comment}




\section{Decision algorithm: scalability, optimality and flexibility}

The decision algoithms runs both in \localControl and \globalControl. In a global level in controller, it performs cross-resource and cross-demand coordination to optimize for quality metrics or cost metrics specified by application level goals. In a local level, it makes real-time decision based on the global decision. In this section, we discuss the design challenges and present our solution to address them. To focus the discussion on decision algorithm, we make two assumptions -- there is only one controller and the \localControl follows the global hints. 

There are two major challenges in designing these algorithms.
\begin{packeditemize}
	\item Scalability: The controller must make decision of multicast trees for thousands of streams and hundreds of server clusters in a relatively short time. It becomes even harder as multicast tree algorithms are known to be time-consuming~\cite{?}.
	\item Optimality: Optimize for quality and cost metrics in a global scale.
	\item Flexibility of multipath: trade-offs between more provisioning (more paths) and data plane performance.
\end{packeditemize}

\subsection{Problem formulation}

\jc{From HotNets}

\subsection{Greedy, iterative algorithm}

\jc{From HotNets}




\section{Merging algorithm: fault tolerance and consistency}


\subsection{Data plane fault tolerance}
Merging algorithm combined the global hints and local hints that are generated by two control loops into one realtime decision. Global hints are global coordination but it has longer latency as it has to go through the remote controllers. In contrast, local hints are made based on (almost) realtime available local information, but less coordinated. Therefore, the merging algorithm should rely more on global hints when network state is stable and rely more on local hints when local hints are too different from global hints which indicates a link failure, a new stream is found, or etc, and the controller needs time to coordinate.

\subsection{Multiple-controller consistency}
Design: Randomly pick the newest hint of each controller and follows its hint.  Key idea: each hint is already guaranteed to have a path to origin




\section{Other issues}

\subsection{Mutiple-version inconsistency}






\section{Prototype Implementation}

We build a prototype of SDCDN that implement all the elements described above. The controller is 

\section{Evaluation}

\subsection{Micro-benchmarks of decision algorithms}

\subsection{Experiment setup}


\input{related}


\section{Conclusion}

{\footnotesize \bibliographystyle{acm}
\bibliography{../common/bibliography}}


\theendnotes

\end{document}







