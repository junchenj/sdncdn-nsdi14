\begin{comment}
\section{Fault tolerance}

We consider two types of failure models. 

\section{Controller consistency}

\begin{itemize}
	\item Inconsistency of multiple versions of same controller: formulate the problem, introduce the solution
	\item Inconsistency of different controllers: formulate the problem, introduce the solution
\end{itemize}
\end{comment}




\section{Decision algorithm: scalability, optimality and flexibility}

The decision algoithms runs both in \localControl and \globalControl. In a global level in controller, it performs cross-resource and cross-demand coordination to optimize for quality metrics or cost metrics specified by application level goals. In a local level, it makes real-time decision based on the global decision. In this section, we discuss the design challenges and present our solution to address them. To focus the discussion on decision algorithm, we make two assumptions -- there is only one controller and the \localControl follows the global hints. 

There are two major challenges in designing these algorithms.
\begin{packeditemize}
	\item Scalability: The controller must make decision of multicast trees for thousands of streams and hundreds of server clusters in a relatively short time. It becomes even harder as multicast tree algorithms are known to be time-consuming~\cite{?}.
	\item Optimality: Optimize for quality and cost metrics in a global scale.
	\item Flexibility of multipath: trade-offs between more provisioning (more paths) and data plane performance.
\end{packeditemize}

\subsection{Problem formulation}

\jc{From HotNets}

\subsection{Greedy, iterative algorithm}

\jc{From HotNets}




\section{Merging algorithm: fault tolerance and consistency}


\subsection{Data plane fault tolerance}
Merging algorithm combined the global hints and local hints that are generated by two control loops into one realtime decision. Global hints are global coordination but it has longer latency as it has to go through the remote controllers. In contrast, local hints are made based on (almost) realtime available local information, but less coordinated. Therefore, the merging algorithm should rely more on global hints when network state is stable and rely more on local hints when local hints are too different from global hints which indicates a link failure, a new stream is found, or etc, and the controller needs time to coordinate.

\subsection{Multiple-controller consistency}
Design: Randomly pick the newest hint of each controller and follows its hint.  Key idea: each hint is already guaranteed to have a path to origin




\section{Other issues}

\subsection{Mutiple-version inconsistency}




